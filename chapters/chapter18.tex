\chapter{线性代数 - 矩阵的转置和逆}

\begin{figure}[ht]
  \centering
  \includegraphics[width=1\linewidth]{asset/20230911220704.png}
\end{figure}

\newpage

我们已经学习过很多关于矩阵的知识点, 今天依然还是矩阵的相关知识. 我们来学一个相关操作「矩阵的转置」, 更重要的是我们需要认识「矩阵的逆」

\section{矩阵的转置}

关于矩阵的转置, 咱们导论课里有提到过. 转置实际上还是蛮简单的, 内容也比较少. 我们来看: 

\begin{align*}
  A = \begin{bmatrix}
  a_{11} \quad a_{12} \quad \cdots \quad a_{1n} \\
  a_{21} \quad a_{22} \quad \cdots \quad a_{2n} \\
  \vdots \quad \vdots \quad \ddots \quad \vdots \\
  a_{n1} \quad a_{n2} \quad \cdots \quad a_{nn} \\
  \end{bmatrix}, \qquad 
  A^T = \begin{bmatrix}
  a_{11} \quad a_{21} \quad \cdots \quad a_{n1} \\
  a_{12} \quad a_{22} \quad \cdots \quad a_{n2} \\
  \vdots \quad \vdots \quad \ddots \quad \vdots \\
  a_{1n} \quad a_{2n} \quad \cdots \quad a_{nn} \\
  \end{bmatrix}
\end{align*}

根据上面我们可以看到, 对于矩阵A来说, 其转置定义为\(A^T\). 而其本质就是行列互换, 我相信一些做数据清洗的小伙伴会经常做转置的操作. 不管是在Excel还是Python中. 

一个矩阵执行转置操作之后, 原来的第一列就变成第一行, 原来第一行变成了第一列. 就相当于沿着对角线的方向去做了一个对称. 

在这里举的例子是一个方阵, 行列数相等, 不是方阵也有转置, 定义也是一样的, 就是沿着对角线方向做一个翻转, 镜面的一个翻转就行了. 

转置也有一些相关性质, 包括$(AB)^T = B^TA^T$, 还有$(A^T)^T = A$. 

也就是如果是AB做一个矩阵乘法, 乘在一起的结果做一个转置, 等于先对B做一个转置再乘上A的转置. 

这个性质其实也不难证明. 首先看行数列数, 比如$A$是n行m列, $B$是m行k列. 乘积的结果应该是n行k列. 做一下转置就是k行n列;$B^T$就是k行m列, $A^T$就是m行n列. 求出来结果也是k行n列. 所以就通过判断矩形的形状也能发现式子是成立的. 

还有第二个性质, 对一个矩阵做一下转置之后, 再做一下转置就又变回原来的矩阵, 很像负负得正的感觉. 就可以理解成负负得正. 

\section{矩阵的逆}

那接下来的这个知识点, 叫做「矩阵的逆」, 或者说一个矩阵的「逆矩阵」. 

一般是对方正去做, 方正是行数等于列数. 我们说对于一个方正$A$, 若存在一个矩阵$B$, 使的:

\begin{align*}
  A \cdot B = B \cdot A = I
\end{align*}

那么我们就称B是A的逆矩阵, 记作$A^{-1}$,  A被称为可逆矩阵;A和B互为逆矩阵. 

那这其中, I是什么呢?I的形状如下: 

\begin{align*}
I = \begin{bmatrix} 1 \quad \cdots \quad 0 \\ \vdots \quad \ddots \quad \vdots \\ 0 \quad \cdots \quad 1 \end{bmatrix}
\end{align*}

那I呢, 我们称之为「\textit{单位矩阵}」. 

那么I呢, 首先是一个方阵, 那就是是说它是n行n列了. 它只有这个主对角线上面元素全部为1, 其他地方的元素全部为0, 我们就把这样的矩阵叫做单位矩阵. 

如果存在这样满足条件的矩阵B,那我们就说B是A的一个逆矩阵, 把它记为$A^{-1}$. 同时A也被称为可逆矩阵, 可逆意思就是能求出来一个逆矩阵. 

逆矩阵它是有一个相互性的, B是A的逆矩阵, 同样的A也是B的逆矩阵. 式子上看, 就是对称的.

我们怎么样判断一个矩阵是否可逆,有个重要条件,就是它必须是满秩的. 其实也就是说, 矩阵可逆与矩阵满秩是一回事, 这两个相当于就是等价的. 

满秩的意思, 就是我有一个方阵n行n列, 之前咱们讲过行秩列秩. 如果它的行秩等于它的行数的话, 我们就说它是满秩的. 比如说n行n列, 它的行秩也等于n, 那也就是说n个行向量全部相互之间都是线性无关的, 那它就称为满秩的. 

矩阵可逆等价于它是满秩的, 满不满秩你怎么样判断?就是当你对矩阵去做行变换或者说列变换的时候, 或者说两个混合的时候, 如果发现做了一些变换之后它有某一行或者某一列全部为零, 那它就不是满秩的, 除此之外它就是一个满秩的. 

逆矩阵还有一个唯一性的定理, 如果矩阵$A$是一个可逆的, 那它逆矩阵不光存在而且是唯一的. 

那怎么样证明唯一性, 这是很多人的一个噩梦, 尤其是对一些文科的同学而言. 

数学上, 很多时候其实不是计算题把人给吓跑了, 而是证明题把人给吓跑了. 搞来搞去我怎么知道用哪一个定理去证明哪一个定理. 所以往往是大家把这些东西想的太复杂, 还有就是你脑海里面在思考的时候, 想出了所有可能用到的定理, 想的越多就越乱, 就越不知道用哪一个. 

相反知道证明题怎么样去做的同学头脑里面其实并没有想太多的东西, 头脑里面可能备选的选择非常有限, 但是这些备选的选择非常清晰. 

对于唯一性的判断有一种小窍门, 就是反正法. 

我就先对着干, 就说不是唯一的, 假设不唯一. 按照假设证明下去, 肯定会有矛盾, 那只要能产生这样矛盾, 就说明假设错了, 其实也就证明出来了. 

好, 我们在这里证明一下「\textit{若$B,$ $C$都是$A$的逆矩阵}」,  则有: $B$和$C$都是$A$的一个逆矩阵, 而且$B$不等于$C$, $B$和$C$是不同的. 

然后我们来证明, $B$它等于$B$乘以单位矩阵($B = B\cdot I$), 这是一个性质, 可以把单位矩阵$I$理解成我们那个数字里面的$1$, 任何数乘以$1$都等于其本身. 大家可以把单位矩阵理解成数字里面的1, 任何矩阵乘它都等于矩阵本身. 

之后就是$I = A \cdot A^{-1}$, $A^{-1}$就是它的逆矩阵, 因为这两个乘在一起就等于单位矩阵. 

接下来是见证奇迹的时刻, $C$是$A$的一个逆矩阵, 所以我把$A^{-1}$次方换成$C$, 这步是按照咱们的假设来的, 没有问题. $A\cdot A^{-1} = A \cdot C$ 

矩阵是符合结合率的, 就是在乘法中, $2\times (3 \times 4)$和$(2 \times 3) \times 4$是一样的, 矩阵的一个结合率也是成立的, 那$(B\cdot A) \cdot C$也就等于$B \cdot (A \cdot C)$. 

然后, $B, A$因为是互为逆矩阵的, 所以相乘一定是得到$I$, 那我们就得到了$I \cdot C$, 之前说了, $I$乘以任何一个矩阵都等于矩阵本身, 那最后就等于$C$. 完整步骤如下:

\begin{align*}
  & B = BI = B(AA^{-1}) = B(AC) = (BA)C = IC = C \\
  & B = C, A\mbox{的逆矩阵唯一}
\end{align*}

证明了半天, 我们发现B是等于C的, 那说明, B和C是相同的, 也就是说, 我们的假设根本不成立, 那就是说A的矩阵它就是唯一的. 

证明完之后, 我们再来看逆矩阵的两个比较常见的性质: 

- $(A^{-1})^{-1} = A$
- $(A^T)^{-1} = (A^{-1})^T$

这里大家就可以类比一下矩阵的转置那一部分的内容, 负负得正. 取一次逆再取一次逆, 就等于它本身. 

然后$A$的转置, 再对转置求一个逆, 就等于先对A求逆, 然后再把逆矩阵转置一下. 这个是不是有点像幂运算?有点像乘方运算对吧?$(x^a)^b$其实就等于$(x^b)^a$, 这是一样的. 

\section{例子: 行操作求矩阵的逆}

我们怎么样去求一个矩阵的逆矩阵?接下来, 咱们看一个例子: 

\textbf{例:求矩阵 $A = \begin{bmatrix} 0 & 1 & 2 \\  1 & 2 & 1 \\ 2 & -1 & 0 \end{bmatrix}$ 的逆矩阵 $A^{-1}$.}

这个涉及多种方法, 有伴随矩阵的方法, 还有我们接下来要用的行操作的方法. 这个方法的本质就是先将矩阵把$A$变化成一个增广矩阵$[A|I]$, 其中$I$是与$A$具有相同维度的单位矩阵. 接下来, 通过一系列的行变化, 将左侧的矩阵$A$转为单位矩阵, 那右侧的$I$就会变为$A$的逆矩阵$A^{-1}$.

\begin{align*}
  B = [A|I] =  
  \begin{bmatrix}
    0 & 1 & 2 & \vline & 1 & 0 & 0\\ 
    1 & 2 & 1 & \vline & 0 & 1 & 0 \\
    2 & -1 & 0 & \vline & 0 & 0 & 1
  \end{bmatrix}
\end{align*}

接下来, 我们就要通过一系列的行操作, 将这个增广矩阵的左侧, 也就是原来的$A$变为单位矩阵. 在变化的同时, 右侧的单位矩阵$I$也要做同样的变化. 我们需要进行行变化, 包括:

\begin{itemize}
  \item 交换行: 通过交换行的顺序来调整矩阵中的行. 
  \item 乘以常数: 用一个非零常数乘以矩阵的某一行. 
  \item 行加减法: 用一个行的倍数加到另一个行上. 
\end{itemize}

我们需要进行一系列操作, 将$A$变为一个单位矩阵, 在原矩阵$A$进行变化的同时, 单位矩阵$I$也要有相同的变化. 操作步骤如下, 左侧为$A$的变化, 右侧为$I$的变化:

\begin{align*}
& \begin{bmatrix}
    0 & 1 & 2 & \vline & 1 & 0 & 0 \\
    1 & 2 & 1 & \vline & 0 & 1 & 0 \\
    2 & -1 & 0 & \vline & 0 & 0 & 1 \\
  \end{bmatrix} 
  \rightarrow
  \begin{bmatrix}
    1 & 2 & 1 & \vline & 0 & 1 & 0 \\
    0 & 1 & 2 & \vline & 1 & 0 & 0 \\
    2 & -1 & 0 & \vline & 0 & 0 & 1 \\
  \end{bmatrix}
  \rightarrow
  \begin{bmatrix}
    1 & 2 & 1 & \vline & 0 & 1 & 0 \\
    0 & 1 & 2 & \vline & 1 & 0 & 0 \\
    0 & -5 & -2 & \vline & 0 & -2 & 1 \\
  \end{bmatrix}  \\
& \rightarrow
  \begin{bmatrix}
    1 & 2 & 1 & \vline & 0 & 1 & 0 \\
    0 & 1 & 2 & \vline & 1 & 0 & 0 \\
    0 & 0 & 8 & \vline & 5 & -2 & 1 \\
  \end{bmatrix} 
  \rightarrow
  \begin{bmatrix}
    1 & 2 & 1 & \vline & 0 & 1 & 0 \\
    0 & 1 & 2 & \vline & 1 & 0 & 0 \\
    0 & 0 & 1 & \vline & \frac{5}{8} & -\frac{1}{4} & \frac{1}{8} \\
  \end{bmatrix}
  \rightarrow
  \begin{bmatrix}
    1 & 2 & 1 & \vline & 0 & 1 & 0 \\
    0 & 1 & 0 & \vline & -\frac{1}{4} & \frac{1}{2} & -\frac{1}{4} \\
    0 & 0 & 1 & \vline & \frac{5}{8} & -\frac{1}{4} & \frac{1}{8} \\
  \end{bmatrix} \\
& \rightarrow
  \begin{bmatrix}
    1 & 2 & 0 & \vline & -\frac{5}{8} & \frac{5}{4} & -\frac{1}{8} \\
    0 & 1 & 0 & \vline & -\frac{1}{4} & \frac{1}{2} & -\frac{1}{4} \\
    0 & 0 & 1 & \vline & \frac{5}{8} & -\frac{1}{4} & \frac{1}{8} \\
  \end{bmatrix}
  \rightarrow
  \begin{bmatrix}
    1 & 0 & 0 & \vline & -\frac{1}{8} & \frac{1}{4} & \frac{3}{8} \\
    0 & 1 & 0 & \vline & -\frac{1}{4} & \frac{1}{2} & -\frac{1}{4} \\
    0 & 0 & 1 & \vline & \frac{5}{8} & -\frac{1}{4} & \frac{1}{8} \\
  \end{bmatrix}
\end{align*}

是不是有点迷糊?没关系,我们下面一步一步的来讲解这是一个怎样的变化过程:

\begin{enumerate}
  \item 将第1行与第2行交换位置, 使第1个非零元素$\mathord{pivot}$出现在第1个位置. 
  \item 用第1行的第1个元素清零第3行的第1个元素, 第3行的第1个元素变为0, 以确保第1个列上的非零元素只出现在第1行. ($-2 \times A_{1n} + A_{3n}$).
  \item 通过将第2行的第2个元素的倍数加到第3行的第2个元素上, 将第3行的第2个元素变为0. ($5 \times A_{2n} + A_{3n}$)
  \item 将第3行除以第3行的第3个元素, 得到1. ($A_{3n} \div 8$)
  \item 通过将第3行的第3个元素的倍数加到第2行的第3个元素上, 将第2行的第3个元素变为0. ($-2 \times A_{3n} + A_{2n}$)
  \item 通过将第3行的第3个元素的倍数加到第1行的第3个元素上, 将第1行的第3个元素变为0. ($ -1 \times A_{3n} + A_{1n}$)
  \item 通过将第2行的第2个元素的倍数加到第1行的第2个元素上, 将第1行的第2个元素变为0. ($ -2 \times A_{2n} + A_{1n}$)
\end{enumerate}

当最终得到第7步里的结果, 我们看到增广矩阵$B$中左边的A矩阵已经变成了单位矩阵, 而这个增广矩阵$B$中右边的$I$已经变的面目全非, 这个最终的变化结果, 就是最初$A$矩阵的逆矩阵$A^{-1}$, 结果就是:
\begin{align*}
A^{-1} = \begin{bmatrix} 
-\frac{1}{8} & \frac{1}{4} & \frac{3}{8} \\
-\frac{1}{4} & \frac{1}{2} & -\frac{1}{4} \\
\frac{5}{8} & -\frac{1}{4} & \frac{1}{8} \\
\end{bmatrix}
\end{align*}

这个就是我们手工计算逆矩阵的一个过程. 

这思路最核心的就是相同的行操作同时作用于这两个矩阵. 那如果我能把$A$变成单位矩阵的话, 也就相当于给$A$乘上了一个$A$的逆矩阵, 那与此同时右边也相当于$I$乘上了一个$A$的逆矩阵. 但是我们知道, I乘上任何一个矩阵其结果都是这个矩阵本身, 所以最终$I$变化的结果, 也就是$A$的逆矩阵了. 

课后呢, 大家可以自己去计算一下, 感受一下. 练习一下对大家去体会和理解还是有好处的. 